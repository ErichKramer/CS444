\documentclass[letterpaper,10pt,notitlepage,fleqn]{article}

\usepackage{alltt}                                           
\usepackage{float}
\usepackage{color}
\usepackage{indentfirst}
\usepackage{url}
\usepackage{balance}
\usepackage{enumitem}
\usepackage{geometry}
\usepackage{hyperref}
\usepackage{textcomp}
\usepackage{listings}
\usepackage{graphicx}
\usepackage{amsfonts}
\usepackage{amsmath}
\usepackage{titling}

\geometry{textheight=8.5in, textwidth=6in}

%random comment

\newcommand{\cred}[1]{{\color{red}#1}}
\newcommand{\cblue}[1]{{\color{blue}#1}}

\newcommand{\toc}{\tableofcontents}

%\usepackage{hyperref}

\title{Handling IO in Linux, FreeBSD and Windows}
\date{2018-05-09}
\author{Erich Kramer}


%pull in the necessary preamble matter for pygments output
\input{pygments.tex}

%% The following metadata will show up in the PDF properties


\parindent = 0.0 in
\parskip = 0.1 in


\begin{document}
\begin{titlepage}
\vspace*{\fill}

\newcommand{\HRule}{\rule{\linewidth}{0.5mm}} % Defines a new command for the horizontal lines, change thickness here

\center % Center everything on the page

%----------------------------------------------------------------------------------------
%TITLE SECTION
%----------------------------------------------------------------------------------------

%\includegraphics[scale=.5]{image.eps}
\HRule \\[0.4cm]
{ \huge \bfseries Writing \#2 \\IOScheduling}\\[0.4cm] % Title of your document

%----------------------------------------------------------------------------------------
%HEADING SECTIONS
%----------------------------------------------------------------------------------------

\textsc{\Large Oregon State University}\\[0.5cm] % Name of your university/college
\textsc{\Large CS444 Operating System II}\\[0.5cm] % Major heading such as course name
\textsc{\large Spring 2018}\\[0.5cm] % Minor heading such as course title
        \noindent \textbf{Erich \textsc{Kramer}} \\ % Your name

\HRule \\[1.5cm]
%----------------------------------------------------------------------------------------
%AUTHOR SECTION
%------------------------------------ --------------------------------------------------
                %----------------------------------------------------------------------------------------
                %DATE SECTION
                %-----------------    -----------------------------------------------------------------------

{\large \today}\\[3cm] % Date, change the \today to a set date if you want to be precise

%----------------------------------------------------------------------------------------
%ABSTRACT SECTION



\vfill % Fill the rest of the page with whitespace



\end{titlepage}

\section{Linux}
%%%%%%%%%%%%%%%%%%%%%%%%%%%%%%%%%%%%%%%%%%%%%%%%%%%%%%%%%%%%%%%%%%%%%%%%%%%%%%%%

%%%%%%%%%%%%%%%%%%%%%%%%%%%%%%%%%%%%%%%%%%%%%%%%%%%%%%%%%%%%%%%%%%%%%%%%%%%%%%%%%%%%%

%%%%%%%%%%%%%%%%%%%%%%%%%%%%%%%%%%%%%%%%%%%%%%%%%%%%%%%%%%%%%%%%%%%%%%%%%%%%%%%%%%%%%


\subsection{IOScheduling}


Linux makes use of three different schedulers, as their applicability varies for 
different devices. It is important to be intelligent and to reorder requests so that those
which are in sequence are read in order. In doing this it is possible to optimize read speed
on average, at the cost of some requests being served out of order. 

\subsubsection{No-Op Scheduler}

The No-Op Scheduler often works best on memory backed block devices which do not have a rotational
system, as optimizing order on these types of devies does not impact overall access time. This would
lead to arbitrary computation and favoritism between processes \cite{stackDisc}. 


\subsubsection{Deadline Scheduler}

The deadline scheduler pushes the Linux Kernel's behavior closer to that of a real time 
operating system. It sets a hard limit on the amount of time I/O might wait, servicing 
those which have exceeded a certain time limit before anything else \cite{stackDisc}. 

\subsubsection{CFQ Scheduler}

To quote the Linux Kernel documentation "The main aim of the CFQ scheduler is to 
provide a fair allocation of the disk I/O bandwidth for all the processes which requests[sic]
an I/O operation." \cite{CFQ}. A process's I/O prority is used to determine how I/O requests
are batched together. The Completely Fair Queuing scheduler 



\section{Windows}
\subsubsection{No-Op Scheduler}

\subsection{IRPs}

Windows uses its own in-house I/O Request Packets (IRPs) for talking to drivers. 




\section{FreeBSD}






\begin{thebibliography}{20}


%LINUX BIB SECTION
    
\bibitem{courseNotesLinux}
    Kevin McGrath, Operating Systems 444 \\
    I/O and provided functionality \\
    Available: Canvas Restricted Access

\bibitem{CFQ}
    Hellwig et. al., Linux Kernel Documentation \\
    Available: https://github.com/torvalds/linux/blob/master/Documentation/block/cfq-iosched.txt

\bibitem{stackDisc}
    haste, StackOverflow discussion
    Available: https://stackoverflow.com/questions/1009577/selecting-a-linux-i-o-scheduler


%WINDOWS BIB SECTION

\bibitem{WinIOView}
    Contributors Microsoft Documentation "Overview of the Windows I/O Model"
    Available: https://docs.microsoft.com/en-us/windows-hardware/drivers/kernel/overview-of-the-windows-i-o-model



%BSD BIB SECTION




\end{thebibliography}

        




\end{document}
