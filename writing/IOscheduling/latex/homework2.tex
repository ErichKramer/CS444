\documentclass[letterpaper,10pt,notitlepage,fleqn]{article}

\usepackage{alltt}                                           
\usepackage{float}
\usepackage{color}
\usepackage{indentfirst}
\usepackage{url}
\usepackage{balance}
\usepackage{enumitem}
\usepackage{geometry}
\usepackage{hyperref}
\usepackage{textcomp}
\usepackage{listings}
\usepackage{graphicx}
\usepackage{amsfonts}
\usepackage{amsmath}
\usepackage{titling}

\geometry{textheight=8.5in, textwidth=6in}

%random comment

\newcommand{\cred}[1]{{\color{red}#1}}
\newcommand{\cblue}[1]{{\color{blue}#1}}

\newcommand{\toc}{\tableofcontents}

%\usepackage{hyperref}

\title{Handling IO in Linux, FreeBSD and Windows}
\date{2018-05-09}
\author{Erich Kramer}


%pull in the necessary preamble matter for pygments output
\input{pygments.tex}

%% The following metadata will show up in the PDF properties


\parindent = 0.0 in
\parskip = 0.1 in


\begin{document}
\begin{titlepage}
\vspace*{\fill}

\newcommand{\HRule}{\rule{\linewidth}{0.5mm}} % Defines a new command for the horizontal lines, change thickness here

\center % Center everything on the page

%----------------------------------------------------------------------------------------
%TITLE SECTION
%----------------------------------------------------------------------------------------

%\includegraphics[scale=.5]{image.eps}
\HRule \\[0.4cm]
{ \huge \bfseries Writing \#2 \\IOScheduling}\\[0.4cm] % Title of your document

%----------------------------------------------------------------------------------------
%HEADING SECTIONS
%----------------------------------------------------------------------------------------

\textsc{\Large Oregon State University}\\[0.5cm] % Name of your university/college
\textsc{\Large CS444 Operating System II}\\[0.5cm] % Major heading such as course name
\textsc{\large Spring 2018}\\[0.5cm] % Minor heading such as course title
        \noindent \textbf{Erich \textsc{Kramer}} \\ % Your name

\HRule \\[1.5cm]
%----------------------------------------------------------------------------------------
%AUTHOR SECTION
%------------------------------------ --------------------------------------------------
                %----------------------------------------------------------------------------------------
                %DATE SECTION
                %-----------------    -----------------------------------------------------------------------

{\large \today}\\[3cm] % Date, change the \today to a set date if you want to be precise

%----------------------------------------------------------------------------------------
%ABSTRACT SECTION



\vfill % Fill the rest of the page with whitespace



\end{titlepage}

\section{Linux}
%%%%%%%%%%%%%%%%%%%%%%%%%%%%%%%%%%%%%%%%%%%%%%%%%%%%%%%%%%%%%%%%%%%%%%%%%%%%%%%%

%%%%%%%%%%%%%%%%%%%%%%%%%%%%%%%%%%%%%%%%%%%%%%%%%%%%%%%%%%%%%%%%%%%%%%%%%%%%%%%%%%%%%

%%%%%%%%%%%%%%%%%%%%%%%%%%%%%%%%%%%%%%%%%%%%%%%%%%%%%%%%%%%%%%%%%%%%%%%%%%%%%%%%%%%%%


\subsection{IOScheduling}

\subsubsection{Deadline Scheduler}


\subsubsection{No-Op Scheduler}



\section{Windows}

\subsection{Processes}

\subsection{Threads}

\subsection{Scheduling}



\section{FreeBSD}

\subsection{Processes}

\subsection{Threads}

\subsection{Scheduling}

\subsubsection{ BSD Scheduler }

\subsubsection{ ULE Scheduler }

\begin{thebibliography}{20}

\bibitem{courseNotesLinux}
    Kevin McGrath, Operating Systems 444 \\
        I/O and provided functionality \\
        [Online] Canvas, Restricted Access

\bibitem{CFQ}
    Hellwig et. al., Linux Kernel Documentation \\
    [Online] https://github.com/torvalds/linux/blob/master/Documentation/block/cfq-iosched.txt


\end{document}
