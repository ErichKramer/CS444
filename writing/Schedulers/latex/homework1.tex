\documentclass[letterpaper,10pt,notitlepage,fleqn]{article}

\usepackage{alltt}                                           
\usepackage{float}
\usepackage{color}
\usepackage{indentfirst}
\usepackage{url}
\usepackage{balance}
\usepackage{enumitem}
\usepackage{geometry}
\usepackage{hyperref}
\usepackage{textcomp}
\usepackage{listings}
\usepackage{graphicx}
\usepackage{amsfonts}
\usepackage{amsmath}
\usepackage{titling}

\geometry{textheight=8.5in, textwidth=6in}

%random comment

\newcommand{\cred}[1]{{\color{red}#1}}
\newcommand{\cblue}[1]{{\color{blue}#1}}

\newcommand{\toc}{\tableofcontents}

%\usepackage{hyperref}

\title{Processes, Threads, Scheduling in Linux, FreeBSD and Windows}
\date{2018-04-15}
\author{Erich Kramer}


%pull in the necessary preamble matter for pygments output
\input{pygments.tex}

%% The following metadata will show up in the PDF properties


\parindent = 0.0 in
\parskip = 0.1 in


\begin{document}
\begin{titlepage}
\vspace*{\fill}

\newcommand{\HRule}{\rule{\linewidth}{0.5mm}} % Defines a new command for the horizontal lines, change thickness here

\center % Center everything on the page

%----------------------------------------------------------------------------------------
%TITLE SECTION
%----------------------------------------------------------------------------------------

%\includegraphics[scale=.5]{image.eps}
\HRule \\[0.4cm]
{ \huge \bfseries Writing \#1 \\Scheduling, Processes, Threads}\\[0.4cm] % Title of your document

%----------------------------------------------------------------------------------------
%HEADING SECTIONS
%----------------------------------------------------------------------------------------

\textsc{\Large Oregon State University}\\[0.5cm] % Name of your university/college
\textsc{\Large CS444 Operating System II}\\[0.5cm] % Major heading such as course name
\textsc{\large Spring 2018}\\[0.5cm] % Minor heading such as course title
        \noindent \textbf{Erich \textsc{Kramer}} \\ % Your name

\HRule \\[1.5cm]
%----------------------------------------------------------------------------------------
%AUTHOR SECTION
%------------------------------------ --------------------------------------------------
                %----------------------------------------------------------------------------------------
                %DATE SECTION
                %-----------------    -----------------------------------------------------------------------

{\large \today}\\[3cm] % Date, change the \today to a set date if you want to be precise

%----------------------------------------------------------------------------------------
%ABSTRACT SECTION



\vfill % Fill the rest of the page with whitespace



\end{titlepage}

\section{Linux}
%%%%%%%%%%%%%%%%%%%%%%%%%%%%%%%%%%%%%%%%%%%%%%%%%%%%%%%%%%%%%%%%%%%%%%%%%%%%%%%%

%%%%%%%%%%%%%%%%%%%%%%%%%%%%%%%%%%%%%%%%%%%%%%%%%%%%%%%%%%%%%%%%%%%%%%%%%%%%%%%%%%%%%

%%%%%%%%%%%%%%%%%%%%%%%%%%%%%%%%%%%%%%%%%%%%%%%%%%%%%%%%%%%%%%%%%%%%%%%%%%%%%%%%%%%%%

\subsection{Processes}

Processes are exposed via files

A processs is a running program. 

Processes exist in a hierarchy that traces back to the init system. (Or scheduler
on some distrobutions if you want to be weird about it). 

Init->runnable->running->ad nasuem->exit

\subsection{Threads}

Threads are treated as processes in Linux, with the exception that they share memory. 


\subsection{Scheduling}



\subsubsection{CFS Scheduler}

The Linux Kernel uses an algorithm called the "Completely Fair Scheduler" It is the successor\\
to the O(1) Scheduler, which boasted a linear time decision making. 


%\subsubsection{The O(1) Scheduler}
%To be implemented in the final write-up. For now please disregard



\section{Windows}

\subsection{Processes}

According to MSDN docs[1], processes are organized under applications, and each process may contain one or more thread. 


[1] https://msdn.microsoft.com/en-us/library/windows/desktop/ms684841(v=vs.85).aspx

\subsection{Threads}


Threads are managed via a scheduling priority value in the range 0-31. Threads which share a priority
are treated the same by the system. Higher priority threads are run in round robin, dropping to lower priority only
as necessary. If a lower priority thread is running and one of a higher value becomes runnable then the latter thread
takes control for a time slice. 


\subsection{Scheduling}

The windows scheduler interacts directly with the threads of processes.


\section{FreeBSD}

\subsection{Processes}

\subsection{Threads}

\subsection{Scheduling}

\subsubsection{ BSD Scheduler }

\subsubsection{ ULE Scheduler }




\end{document}
