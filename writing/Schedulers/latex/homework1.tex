\documentclass[letterpaper,10pt,notitlepage,fleqn]{article}

\usepackage{alltt}                                           
\usepackage{float}
\usepackage{color}
\usepackage{indentfirst}
\usepackage{url}
\usepackage{balance}
\usepackage{enumitem}
\usepackage{geometry}
\usepackage{hyperref}
\usepackage{textcomp}
\usepackage{listings}
\usepackage{graphicx}
\usepackage{amsfonts}
\usepackage{amsmath}
\usepackage{titling}

\geometry{textheight=8.5in, textwidth=6in}

%random comment

\newcommand{\cred}[1]{{\color{red}#1}}
\newcommand{\cblue}[1]{{\color{blue}#1}}

\newcommand{\toc}{\tableofcontents}

%\usepackage{hyperref}

\title{Processes, Threads, Scheduling in Linux, FreeBSD and Windows}
\date{2018-04-18}
\author{Erich Kramer}


%pull in the necessary preamble matter for pygments output
\usepackage{fancyvrb}
\usepackage{color}
\usepackage[latin1]{inputenc}


\makeatletter
\def\PY@reset{\let\PY@it=\relax \let\PY@bf=\relax%
    \let\PY@ul=\relax \let\PY@tc=\relax%
    \let\PY@bc=\relax \let\PY@ff=\relax}
\def\PY@tok#1{\csname PY@tok@#1\endcsname}
\def\PY@toks#1+{\ifx\relax#1\empty\else%
    \PY@tok{#1}\expandafter\PY@toks\fi}
\def\PY@do#1{\PY@bc{\PY@tc{\PY@ul{%
    \PY@it{\PY@bf{\PY@ff{#1}}}}}}}
\def\PY#1#2{\PY@reset\PY@toks#1+\relax+\PY@do{#2}}

\expandafter\def\csname PY@tok@gd\endcsname{\def\PY@tc##1{\textcolor[rgb]{0.63,0.00,0.00}{##1}}}
\expandafter\def\csname PY@tok@gu\endcsname{\let\PY@bf=\textbf\def\PY@tc##1{\textcolor[rgb]{0.50,0.00,0.50}{##1}}}
\expandafter\def\csname PY@tok@gt\endcsname{\def\PY@tc##1{\textcolor[rgb]{0.00,0.25,0.82}{##1}}}
\expandafter\def\csname PY@tok@gs\endcsname{\let\PY@bf=\textbf}
\expandafter\def\csname PY@tok@gr\endcsname{\def\PY@tc##1{\textcolor[rgb]{1.00,0.00,0.00}{##1}}}
\expandafter\def\csname PY@tok@cm\endcsname{\let\PY@it=\textit\def\PY@tc##1{\textcolor[rgb]{0.25,0.50,0.50}{##1}}}
\expandafter\def\csname PY@tok@vg\endcsname{\def\PY@tc##1{\textcolor[rgb]{0.10,0.09,0.49}{##1}}}
\expandafter\def\csname PY@tok@m\endcsname{\def\PY@tc##1{\textcolor[rgb]{0.40,0.40,0.40}{##1}}}
\expandafter\def\csname PY@tok@mh\endcsname{\def\PY@tc##1{\textcolor[rgb]{0.40,0.40,0.40}{##1}}}
\expandafter\def\csname PY@tok@go\endcsname{\def\PY@tc##1{\textcolor[rgb]{0.50,0.50,0.50}{##1}}}
\expandafter\def\csname PY@tok@ge\endcsname{\let\PY@it=\textit}
\expandafter\def\csname PY@tok@vc\endcsname{\def\PY@tc##1{\textcolor[rgb]{0.10,0.09,0.49}{##1}}}
\expandafter\def\csname PY@tok@il\endcsname{\def\PY@tc##1{\textcolor[rgb]{0.40,0.40,0.40}{##1}}}
\expandafter\def\csname PY@tok@cs\endcsname{\let\PY@it=\textit\def\PY@tc##1{\textcolor[rgb]{0.25,0.50,0.50}{##1}}}
\expandafter\def\csname PY@tok@cp\endcsname{\def\PY@tc##1{\textcolor[rgb]{0.74,0.48,0.00}{##1}}}
\expandafter\def\csname PY@tok@gi\endcsname{\def\PY@tc##1{\textcolor[rgb]{0.00,0.63,0.00}{##1}}}
\expandafter\def\csname PY@tok@gh\endcsname{\let\PY@bf=\textbf\def\PY@tc##1{\textcolor[rgb]{0.00,0.00,0.50}{##1}}}
\expandafter\def\csname PY@tok@ni\endcsname{\let\PY@bf=\textbf\def\PY@tc##1{\textcolor[rgb]{0.60,0.60,0.60}{##1}}}
\expandafter\def\csname PY@tok@nl\endcsname{\def\PY@tc##1{\textcolor[rgb]{0.63,0.63,0.00}{##1}}}
\expandafter\def\csname PY@tok@nn\endcsname{\let\PY@bf=\textbf\def\PY@tc##1{\textcolor[rgb]{0.00,0.00,1.00}{##1}}}
\expandafter\def\csname PY@tok@no\endcsname{\def\PY@tc##1{\textcolor[rgb]{0.53,0.00,0.00}{##1}}}
\expandafter\def\csname PY@tok@na\endcsname{\def\PY@tc##1{\textcolor[rgb]{0.49,0.56,0.16}{##1}}}
\expandafter\def\csname PY@tok@nb\endcsname{\def\PY@tc##1{\textcolor[rgb]{0.00,0.50,0.00}{##1}}}
\expandafter\def\csname PY@tok@nc\endcsname{\let\PY@bf=\textbf\def\PY@tc##1{\textcolor[rgb]{0.00,0.00,1.00}{##1}}}
\expandafter\def\csname PY@tok@nd\endcsname{\def\PY@tc##1{\textcolor[rgb]{0.67,0.13,1.00}{##1}}}
\expandafter\def\csname PY@tok@ne\endcsname{\let\PY@bf=\textbf\def\PY@tc##1{\textcolor[rgb]{0.82,0.25,0.23}{##1}}}
\expandafter\def\csname PY@tok@nf\endcsname{\def\PY@tc##1{\textcolor[rgb]{0.00,0.00,1.00}{##1}}}
\expandafter\def\csname PY@tok@si\endcsname{\let\PY@bf=\textbf\def\PY@tc##1{\textcolor[rgb]{0.73,0.40,0.53}{##1}}}
\expandafter\def\csname PY@tok@s2\endcsname{\def\PY@tc##1{\textcolor[rgb]{0.73,0.13,0.13}{##1}}}
\expandafter\def\csname PY@tok@vi\endcsname{\def\PY@tc##1{\textcolor[rgb]{0.10,0.09,0.49}{##1}}}
\expandafter\def\csname PY@tok@nt\endcsname{\let\PY@bf=\textbf\def\PY@tc##1{\textcolor[rgb]{0.00,0.50,0.00}{##1}}}
\expandafter\def\csname PY@tok@nv\endcsname{\def\PY@tc##1{\textcolor[rgb]{0.10,0.09,0.49}{##1}}}
\expandafter\def\csname PY@tok@s1\endcsname{\def\PY@tc##1{\textcolor[rgb]{0.73,0.13,0.13}{##1}}}
\expandafter\def\csname PY@tok@sh\endcsname{\def\PY@tc##1{\textcolor[rgb]{0.73,0.13,0.13}{##1}}}
\expandafter\def\csname PY@tok@sc\endcsname{\def\PY@tc##1{\textcolor[rgb]{0.73,0.13,0.13}{##1}}}
\expandafter\def\csname PY@tok@sx\endcsname{\def\PY@tc##1{\textcolor[rgb]{0.00,0.50,0.00}{##1}}}
\expandafter\def\csname PY@tok@bp\endcsname{\def\PY@tc##1{\textcolor[rgb]{0.00,0.50,0.00}{##1}}}
\expandafter\def\csname PY@tok@c1\endcsname{\let\PY@it=\textit\def\PY@tc##1{\textcolor[rgb]{0.25,0.50,0.50}{##1}}}
\expandafter\def\csname PY@tok@kc\endcsname{\let\PY@bf=\textbf\def\PY@tc##1{\textcolor[rgb]{0.00,0.50,0.00}{##1}}}
\expandafter\def\csname PY@tok@c\endcsname{\let\PY@it=\textit\def\PY@tc##1{\textcolor[rgb]{0.25,0.50,0.50}{##1}}}
\expandafter\def\csname PY@tok@mf\endcsname{\def\PY@tc##1{\textcolor[rgb]{0.40,0.40,0.40}{##1}}}
\expandafter\def\csname PY@tok@err\endcsname{\def\PY@bc##1{\setlength{\fboxsep}{0pt}\fcolorbox[rgb]{1.00,0.00,0.00}{1,1,1}{\strut ##1}}}
\expandafter\def\csname PY@tok@kd\endcsname{\let\PY@bf=\textbf\def\PY@tc##1{\textcolor[rgb]{0.00,0.50,0.00}{##1}}}
\expandafter\def\csname PY@tok@ss\endcsname{\def\PY@tc##1{\textcolor[rgb]{0.10,0.09,0.49}{##1}}}
\expandafter\def\csname PY@tok@sr\endcsname{\def\PY@tc##1{\textcolor[rgb]{0.73,0.40,0.53}{##1}}}
\expandafter\def\csname PY@tok@mo\endcsname{\def\PY@tc##1{\textcolor[rgb]{0.40,0.40,0.40}{##1}}}
\expandafter\def\csname PY@tok@kn\endcsname{\let\PY@bf=\textbf\def\PY@tc##1{\textcolor[rgb]{0.00,0.50,0.00}{##1}}}
\expandafter\def\csname PY@tok@mi\endcsname{\def\PY@tc##1{\textcolor[rgb]{0.40,0.40,0.40}{##1}}}
\expandafter\def\csname PY@tok@gp\endcsname{\let\PY@bf=\textbf\def\PY@tc##1{\textcolor[rgb]{0.00,0.00,0.50}{##1}}}
\expandafter\def\csname PY@tok@o\endcsname{\def\PY@tc##1{\textcolor[rgb]{0.40,0.40,0.40}{##1}}}
\expandafter\def\csname PY@tok@kr\endcsname{\let\PY@bf=\textbf\def\PY@tc##1{\textcolor[rgb]{0.00,0.50,0.00}{##1}}}
\expandafter\def\csname PY@tok@s\endcsname{\def\PY@tc##1{\textcolor[rgb]{0.73,0.13,0.13}{##1}}}
\expandafter\def\csname PY@tok@kp\endcsname{\def\PY@tc##1{\textcolor[rgb]{0.00,0.50,0.00}{##1}}}
\expandafter\def\csname PY@tok@w\endcsname{\def\PY@tc##1{\textcolor[rgb]{0.73,0.73,0.73}{##1}}}
\expandafter\def\csname PY@tok@kt\endcsname{\def\PY@tc##1{\textcolor[rgb]{0.69,0.00,0.25}{##1}}}
\expandafter\def\csname PY@tok@ow\endcsname{\let\PY@bf=\textbf\def\PY@tc##1{\textcolor[rgb]{0.67,0.13,1.00}{##1}}}
\expandafter\def\csname PY@tok@sb\endcsname{\def\PY@tc##1{\textcolor[rgb]{0.73,0.13,0.13}{##1}}}
\expandafter\def\csname PY@tok@k\endcsname{\let\PY@bf=\textbf\def\PY@tc##1{\textcolor[rgb]{0.00,0.50,0.00}{##1}}}
\expandafter\def\csname PY@tok@se\endcsname{\let\PY@bf=\textbf\def\PY@tc##1{\textcolor[rgb]{0.73,0.40,0.13}{##1}}}
\expandafter\def\csname PY@tok@sd\endcsname{\let\PY@it=\textit\def\PY@tc##1{\textcolor[rgb]{0.73,0.13,0.13}{##1}}}

\def\PYZbs{\char`\\}
\def\PYZus{\char`\_}
\def\PYZob{\char`\{}
\def\PYZcb{\char`\}}
\def\PYZca{\char`\^}
\def\PYZam{\char`\&}
\def\PYZlt{\char`\<}
\def\PYZgt{\char`\>}
\def\PYZsh{\char`\#}
\def\PYZpc{\char`\%}
\def\PYZdl{\char`\$}
\def\PYZti{\char`\~}
% for compatibility with earlier versions
\def\PYZat{@}
\def\PYZlb{[}
\def\PYZrb{]}
\makeatother


%% The following metadata will show up in the PDF properties


\parindent = 0.0 in
\parskip = 0.1 in


\begin{document}
\begin{titlepage}
\vspace*{\fill}

\newcommand{\HRule}{\rule{\linewidth}{0.5mm}} % Defines a new command for the horizontal lines, change thickness here

\center % Center everything on the page

%----------------------------------------------------------------------------------------
%TITLE SECTION
%----------------------------------------------------------------------------------------

%\includegraphics[scale=.5]{image.eps}
\HRule \\[0.4cm]
{ \huge \bfseries Writing \#1 \\Scheduling, Processes, Threads}\\[0.4cm] % Title of your document

%----------------------------------------------------------------------------------------
%HEADING SECTIONS
%----------------------------------------------------------------------------------------

\textsc{\Large Oregon State University}\\[0.5cm] % Name of your university/college
\textsc{\Large CS444 Operating System II}\\[0.5cm] % Major heading such as course name
\textsc{\large Spring 2018}\\[0.5cm] % Minor heading such as course title
        \noindent \textbf{Erich \textsc{Kramer}} \\ % Your name

\HRule \\[1.5cm]
%----------------------------------------------------------------------------------------
%AUTHOR SECTION
%------------------------------------ --------------------------------------------------
                %----------------------------------------------------------------------------------------
                %DATE SECTION
                %-----------------    -----------------------------------------------------------------------

{\large \today}\\[3cm] % Date, change the \today to a set date if you want to be precise

%----------------------------------------------------------------------------------------
%ABSTRACT SECTION



\vfill % Fill the rest of the page with whitespace



\end{titlepage}

\section{Linux}
%%%%%%%%%%%%%%%%%%%%%%%%%%%%%%%%%%%%%%%%%%%%%%%%%%%%%%%%%%%%%%%%%%%%%%%%%%%%%%%%

%%%%%%%%%%%%%%%%%%%%%%%%%%%%%%%%%%%%%%%%%%%%%%%%%%%%%%%%%%%%%%%%%%%%%%%%%%%%%%%%%%%%%

%%%%%%%%%%%%%%%%%%%%%%%%%%%%%%%%%%%%%%%%%%%%%%%%%%%%%%%%%%%%%%%%%%%%%%%%%%%%%%%%%%%%%

\subsection{Processes}

Linux processes exist in memory and have related files resting under /proc so that programs
may interact with other processes in a safe way. Here information about file descriptors, cwd,
attributes and various other important process data can be parsed. Prior to this the solution 
was to allow executing programs access to the Kernel memory data structures \cite{FHS}. 
This has both heavy over head and potential security holes. 


Processes exist in a hierarchy that traces back to the init system. All processes have
a parent which can be either the process which it descends from, or the init system itself.
Typically a process which has the init system as its parent and was created after the boot process
is a daemon\cite{LinuxBook}. In some systems the parent of the init system is PID 0, the 
scheduler. However this is inconsistant and an ambiguous phrasing as it becomes difficult to
distinguish between a NUll value and a PID of 0. 

Typically processes on Linux exist in a few possible states, first they are initialized and 
immedietly are runnable. When they are picked up by the scheduler and ran they may either exit
if they have completed their task or they will enter the blocked state if they exceed their 
allocated CPU time. From there they proceed to runnable ad. nasuem until their termination.
However, before they can be cleaned up they are in a zombie state in which a parent must 
wait on them to die. \cite{McGrath} 


\subsection{Threads}


Prior to Linux 2.4 threads had unique PIDs and were implemented as processes. The main difference 
between them and another system process was that they shared an executable and memory. They were 
still distinct from fork()'d processes. 

However, after Linux2.4 thread groups were added as a feature\cite{ThreadAsProc}. In conjunction 
with the clone() function threads no longer have seperates PIDs. 


\subsection{Scheduling}



\subsubsection{CFS Scheduler}

The Linux Kernel uses an algorithm called the "Completely Fair Scheduler" It is the successor\\
to the O(1) Scheduler, which boasted a linear time decision making. The CFS makes use of a red-black
tree to keep track of used virtual time and assert fairness. The intent is not to assign CPU time
in a round robin fashion equally to processes. It is instead to assign time proportional to the
priority of a process\cite{LinuxSched}. In doing this the scheduler is able to assign time to low 
priority tasks despite while still giving the majority to those of a higher priority. 


%\subsubsection{The O(1) Scheduler}
%To be implemented in the final write-up. For now please disregard



\section{Windows}

\subsection{Processes}

According to MSDN docs\cite{MSProc}, processes are organized under applications, and each 
process may contain one or more thread. These processes have their own security context and
environment, and inside this enviornment there exists a primary thread which instantiates 
execution. From this thread, similarly to the init process in Linux with regards to processes, 
spawns more threads inside the context of this application. This then allows an application to 
apply multiprocessing solutions and so on. 

Of note is also the Job object which exists on windows. As processes manage threads that 
they contain, job objects also manage processes which they contain suggesting even more 
abstraction on windows \cite{MSAbout}. 

\subsection{Threads}

As discussed above threads on windows exist for all applications \cite{MSAbout}. 
Each thread in Windows maintains its own exception handlers, priority and thread local
storage and identifier. As Windows seems to have a fetish for layers of abstraction on execution
partitioning there also exists a layer underneath threads through the fiber concept that MS 
uses. They can be used as a substitute for threads and may be manually scheduled by the program. 


\subsection{Scheduling}

Threads are managed via a scheduling priority value in the range 0-31. Threads 
which share a priority are treated the same by the system. Higher priority threads 
are run in round robin, dropping to lower priority onlyas necessary. If a 
lower priority thread is running and one of a higher value becomes runnable 
then the latter thread takes control for a time slice. \cite{MSScheduling} 
As intuition suggests this is not the best scheduling algorithm for a system and falls 
victim to low priority threads being given no execution time. This is not ideal as while you 
do want to prioritize some tasks you also want to have equality and all processes should be
given CPU time. High load and high priority tasks even risk freezing a system. 



\section{FreeBSD}

\subsection{Processes}

FreeBSD also uses Process IDs as does Linux. There are many similarities between the two 
kernels and their handling of processes as they are both direct Unix derivatives \cite{BSDPgen}. 
Similar to Linux processes are all spawned by the fork() command on one level or another. Often 
features implemented in either Linux or BSD end up becoming uniform between the two as there is 
heavy cross contamination.


\subsection{Threads}

As would be expected FreeBSD implements threads in ways similar to Linux as it sources
itself in POSIX and UNIX specs. Whether the model uses a 1:1 or 1:N process to thread model
is genearlly dependent on the release of BSD and the Kernel customization by users\cite{BSDThread}. 

\subsection{Scheduling}

%\subsubsection{ BSD Scheduler }
\subsubsection{ ULE Scheduler }
The ULE Scheduler is the successor to the BSD scheduler, and was introduced in BSD 5.0 
\cite{BSDProc}. It is currently the default scheduler due to advantages it has over the
BSD Scheduler. Namely it overcame deficites in SMP and SMT. It is comprised of a High 
and Low level Scheduler. The Low level scheduler \cite{ULE} reads from the runqueues and 
places threads into execution, while the high level scheduler operates every few seconds and
arranges the threads to be executed\cite{BSDProc}. 

This scheduler also has a unique Sleepqueue where a program is stored if it is blocking. The 
high level scheduler must operate on threads to move them into priorities.  


\begin{thebibliography}{50}


%\section{Linux}
%Linux Citations

\bibitem{FHS}
    \textit{Filesystem Hierarchy Standard}  The Linux Foundation 

\bibitem{LinuxBook}
    Michael Kerrisk, \textit{The Linux Programming Interface} \\
        No Starch Press San Francisco, CA, USA 2010

\bibitem{McGrath}
    Kevin McGrath Operating Systems 2 Lecture, Oregon State University

\bibitem{ThreadAsProc}
    ikkachu "Are thread implemented as processes on Linux?" [Online] \\
    https://unix.stackexchange.com/questions/364660/are-threads-implemented-as-processes-on-linux

\bibitem{LinuxThreadBlog}
    Himanshu Arora "Introduction to Linux Threads" [Online]\\
    Available: https://www.thegeekstuff.com/2012/03/linux-threads-intro/?utm\_source=tuicool

\bibitem{LinuxSched}
    Lie Ryan "Difference between Windows and Linux Scheduler" Access April 14th 2018 [Online]
        Available: https://superuser.com/questions/414604/difference-between-the-windows-and-linux-thread-scheduler 


%Microsoft Citations

\bibitem{MSProc}
    "Processes and Threads"
        Microsoft Documentation [Online] \\Available:
        https://msdn.microsoft.com/en-us/library/windows/desktop/ms684841(v=vs.85).aspx

\bibitem{MSAbout}
    "About Threads and Processes" Microsoft Documentation [Online] \\Available:
    https://superuser.com/questions/414604/difference-between-the-windows-and-linux-thread-scheduler


\bibitem{MSScheduling}
    "Scheduling Priorities"
        Microsoft Documentation [Online] \\Available:
https://msdn.microsoft.com/en-us/library/windows/desktop/ms685100(v=vs.85).aspx

%BSD Citations

\bibitem{BSDCompare}
    "Comparison of Solaris, Linux, and FreeBSD Kernels" Open Solaris [Online]\\
        Available: https://web.archive.org/web/20080807124435/http://cn.opensolaris.org/files/solaris\_linux\_bsd\_cmp.pdf

\bibitem{BSDProc}
    "Process Management in the FreeBSD Operating System" informIT \\

\bibitem{BSDPgen}
    Processes and Daemons [Online]\\ Available: 
    https://www.freebsd.org/doc/handbook/basics-processes.html

\bibitem{BSDThread}
    "A look inside..." FreeBSD [Online]\\
        Available: https://www.freebsd.org/doc/en\_US.ISO8859-1/articles/linux-emulation/inside.html

\bibitem{ULE}
    \textit{ULE: A Modern Scheduler for FreeBSD} Jeff Roberson The FreeBSD Project
    September 2003

\end{thebibliography}

\end{document}
